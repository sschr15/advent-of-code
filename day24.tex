%! Author = sschr15
%! Date = 12/24/2023

% Preamble
\documentclass[11pt]{article}

% Packages
\usepackage{amsmath,amsfonts}
\usepackage[margin=0.5in]{geometry}

% Document
\begin{document}

\section{Advent of Code 2023 - Day 24}
\label{sec:advent-of-code-2023---day-24}

Today's challenge is to break up hail disrupting the newly-restored snow production.

Each hail stone is defined in the input as \texttt{x, y, z @ a, b, c} where \texttt{x, y, z} are the coordinates of
the stone and \texttt{a, b, c} are the velocity of the stone in the \texttt{x, y, z} directions respectively.
Each number is an integer.

If not constrained to the X-Y plane, the hail stones would never intersect.

\subsection{Part 1}
\label{subsec:part-1}

The challenge for Part 1 of AoC 2023 Day 24 is to find every possible pair of hail stones that, if constrained to
the X-Y plane, would hit each other at some point in the future.
Each hail stone can be represented as a ray in 2D space, and the intersection of two rays can be found by solving the
system of equations and ensuring that the solution is along both rays.
By representing each stone's travel as a ray, it ensures that the hail stones intersecting in the past are not
counted.
The answer to Part 1 is the number of pairs of rays that intersect.

\subsection{Part 2}
\label{subsec:part-2}

The second part of the challenge is to find a single trajectory and starting location that will hit every hail stone
    at some point.
I found no simple programmatic solution to this problem, but I \textit{do} know the set of equations that must be solved
    to find the solution.

\subsubsection{Equations}
\label{subsubsec:equations}

Suppose a hail stone can be defined as $h_i$ for $i \in \{1, 2, \dots, n\}$, and the trajectory of the hail stone
    can be defined as some function $h_i(t)$ for $t \in \mathbb{W}$.
The trajectory of the rock, $r(t)$ must intersect with every hail stone at some point in time, so the following
    equations must be satisfied:

\begin{align*}
    \forall i \in \{1, 2, \dots, n\}, \exists t_i \in \mathbb{W} \text{ such that } r(t_i) = h_i(t_i)
\end{align*}

\subsubsection{Solution}
\label{subsubsec:solution}

With information from other users online, I used Microsoft's Z3 Theorem Prover to solve this system of equations.
By defining a starting position and velocity for the thrown rock and a $t_i$ for each hail stone's time of impact, each
    component of the equation can be defined as a constraint.
One final constraint is added for each $t_i$ to ensure it is non-negative, i.e.\ the hail stone is hit in the future.

\end{document}
